\documentclass[]{report}
\usepackage{url}
\usepackage{graphicx}
\usepackage{amsmath,amssymb}
\usepackage[numbers]{natbib}
\usepackage{titlesec}

% Title Page
\title{Evaluation Report}
\author{Liam Brand}
\date{}


\begin{document}
\maketitle

\section{The System Produced}
The overall project aim was to develop a prototype system that involves several sensors monitoring an environment, and then using the data acquired from this for analytics and visualization through an appropriate medium. This section aims to explore the implemented features with regards to how well they were implemented according to various criteria.

	\subsection{Fitness for Purpose}
	%The system meets the overall project aim explained earlier. Several sensors have been successfully developed that take environmental data such as temperature and humidity, and this data is sent to a hosted database that stores these readings. A front end website has been developed and hosted as shown in the product demo, and among other things the website features data visualization which comes in the form of charts providing a visual breakdown of different environmental aspects, as well as elements that display information gained from data analysis such as the temperature being too hot.
	Subsystems were successfully implemented to meet the overall project goal outlined earlier, with two sensors being developed that post data to a database and a hosted front end containing appropriate functionality for data visualization and analysis. The physical demo was proof of the product's functionality.
	
	As well, these elements have been implemented to be near real-time, as stored and processed data needs to be up to date for the system to be accurate. Analysing an hour old reading for example might not provide useful information as it's uncertain whether or not this reading represent the environment's current state. The sensors post their readings to the database every 15 seconds, allowing for the database to receive sensory data at a relatively quick pace. As well, XML http requests have been configured to fire every few seconds for relevant data visualization and data analysis methods, ensuring what is displayed or analysed is the most up to date information available and provides the most value to the user.
	
	As of right now the visualization and analysis subsystems are hard coded to only retrieve data from the two created devices, moving forward this should be made more dynamic to accommodate for additional created devices to allow these subsystems to function properly as more devices are added.
	
	\subsection{Robustness}
	AWS Lambda is used, which features functionality to allow scaling\cite{awslambdadocs} allowing the database to cope with additional device implementations. The devices themselves have features to cope with failures. It features two temperatures sensors with the second one being used if the first one fails, features in code to prevent erroneous measurements. As well, the devices will automatically attempt to reestablish internet connection upon being disconnected, and they also feature a 32GB Micro-SD card as temporary storage for readings data should connection be lost. Web elements such as the model's code contains try/catch statements to handle exceptions such as empty readings from the database as well, and will print appropriate messages to the console to assist in troubleshooting should these issues occur.
	
	\subsection{Look and Feel}
	Petrie and Fraser\cite{petrie2004tension} describe many different issues that significantly impede users' ability to properly make use of a website. The primary offenders here are complex pages, unclear navigation and poor colour contrast. The website's pages are very simple, with large block elements for listed devices and clear headings denoting sections such as the intelligent model's device comparison results. Navigation is achieved via the use of a simple navbar, something seen very often in websites and shouldn't be unfamiliar to users that have used other websites before. As well, the website's colour scheme is mainly dark blue and white, offering a clear contrast between the background and elements such as text. These elements allow the system's interactive element to be intuitive, familiar, easy to use and visually pleasing.
	
	\subsection{Consistency}
	Subsystems have been developed with other subsystems in mind. For example, the website front end was properly formatted to make space for the data visualization and data analytics. These elements have clear areas of the web page reserved for them that allows them to blend into the rest of the website, and they were included in the media queries to allow the website to scale to different screen sizes. The database features endpoints allowing the developed sensors to post data to the database, and endpoints were also created to provide the visualization and analysis subsystems with the sensory data that they needed to perform their functions. These things allowed the subsystems to work together without anything feeling out of place or not accommodated for.
	
	\subsection{Technical Evaluation}
	
	
	\subsection{Financial Breakdown}
	As part of the model's development, the system's implementation needs to be financially evaluated to determine the cost of its deployment on an industrial level. The Terms of Reference outlines the project's main finances, but here will be a more in depth exploration of the financial aspects behind the model specifically.


\section{Project Management, Process and Personal Achievement}
\subsection{Requirements Gathering}
Requirements were gathered through the use of client questionnaires and interviews. Whilst the overall requirements of the system were already known, this allowed us to answer some questions about finer details that came up during the initial planning and distribution of work. I encountered a missed opportunity for some important information with the questionnaires however. An element of my subsystem was using weather data to help gauge whether or not the client's house temperature was comfortable, but it would have been useful to have some input on precisely how this would be done. For example, following an external temperature spike/drop does the client want the heating instantly adjusted or done slowly over a period of time to prevent it feeling uncomfortable? Do they perhaps want to be able to opt out of this small feature due to the potential inconsistency of weather forecasts and simply fine tune their house temperature themselves?

\section{Professional Issues}


\section{Legal Issues}


\section{Social Issues}


\section{Ethical Issues}

	\bibliographystyle{plainnat}
	\bibliography{papers}
\end{document}          
